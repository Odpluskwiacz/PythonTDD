\documentclass{article}
\usepackage{graphicx} % Required for inserting images
\usepackage{float} % Pakiet pozwalający na użycie specyfikatora H
\usepackage[polish]{babel} % Polish language support
\usepackage[T1]{fontenc} % Font encoding
\usepackage[utf8]{inputenc} % To ensure UTF-8 encoding
\usepackage{geometry} % For page layout
\geometry{a4paper, margin=1in} % Set page size and margins
\usepackage{minted} % Wstawianie sformatowanego Kodu
\usepackage{hyperref}  % Pakiet do tworzenia odnośników


\title{Inżynieria Oprogramowania - Projekt Aplikacji}
\author{Michal.bralski , Stanisław Kuzia }
\date{May 2025}

\begin{document}

\maketitle

\section{Wprowadzenie}
Celem projektu było stworzenie poglądowego działania między planetarna mobilnej stacji odwiertniczej, której zadaniem jest wydobycie zlokalizowanie oraz wysyłka surowców 

\section{Struktury aplikacji}
Aplikacja składa się z trzech głownych klas
\begin{enumerate}
    \item AGI - Artificial General Inteligence
    \item Corpo - Odpowiada za manualne nadpisanie komendy wysłania surowców oraz pobiera dane z sądy odkrywczej w celach komercyjnych
    \item Militaria - Odpowiada za detonację ładunku strategicznego oraz pobiera dane z sądy odkrywczej
\end{enumerate}

\section{Funkcjonalność}
Główne założenia systemu wspomaganego syntetycznym rozumowaniem.
    \begin{enumerate}
        \item Dokonywanie odwiertów w celu pozyskania złóż i zadowoleniu udziałowców.
        \item fabrykowanie sąd w celach naukowo-komercyjnej eksploracji przestrzeni kosmicznej.
        \item Poszukiwanie nowych złóż strategicznych ku dobrobycie rasy ludzkiej.
        \item Logistyczne przygotowanie oraz wysyłka surowców nierafinerowanych
        \item Strategiczne eliminowanie celów xenologicznych.
    \end{enumerate}


\section{Wybrane funkcje i ich testy}

\subsubsection{drillStatus\{Bool\}}
\begin{enumerate}
    \item CheckTemp
    \item CheckRotation
    \item CheckHardnes
    \item CheckMandatoryMaitanceDate
\end{enumerate}

\subsubsection{FabricatorStatus\{float\}}
\begin{enumerate}
    \item CheckMandatoryMaitanceDate
    \item CheckAutonomusWeelding
    \item CheckOreDeposit
    \item CheckFabricatorFailSafe
    \item CheckFumeeExhaus
\end{enumerate}

\subsubsection{TelemetryStatus\{bool\}}
\begin{enumerate}
    \item CheckInterstalarLink
    \item CheckTeleRangeLink
    \item CheckInsiders6GConection
    \item CheckModbusStatus
    \item CheckPlanetarLauncher
\end{enumerate}

\subsubsection{TacticialMisleTest\{bool\}}

\begin{enumerate}
    \item CheckUraniumHalf-Life
    \item CalibrateMissle
    \item CheckTruster
    \item FabricateMissingMissiles
    \item DisposeDettiretedCores
\end{enumerate}

\subsubsection{LogisticDepartmentStatus\{int\}}

\begin{itemize}
    \item HaulingEquipmentStatus
    \item ScrapYardCappStatus
    \item ScrappingFurnaceStatus
    \item ScrappingBeaconStatus
    \item RaylCannoStatus
\end{itemize}


\section{Diagram UML}

\begin{figure}[H]
\centering
\includegraphics[width=1\linewidth]{Przypadkow.png}
\caption{\label{fig:1} Diagram Przypadków Użycia}
\end{figure}

\begin{figure}[H]
\centering
\includegraphics[width=1\linewidth]{Aktywnosc.png}
\caption{\label{fig:2} Diagram Aktywności}
\end{figure}

\begin{figure}[H]
\centering
\includegraphics[width=1\linewidth]{Sekwencji.png}
\caption{\label{fig:1} Diagram Sekwencji}
\end{figure}

\section{Wnioski}
Zrealizowany projekt umożliwił praktyczne zastosowanie podstawowych
zagadnień związanych z programowaniem oraz testowaniem jednostkowym.
Struktura kodu została zaprojektowana w sposób modularny, co ułatwia jego dalszą rozbudowę i
utrzymanie. Dodatkowo, przygotowane testy jednostkowe pomogły w weryfikacji
poprawności działania kluczowych metod oraz umożliwiły wychwycenie ewentualnych błędów
na etapie tworzenia aplikacji.

\end{document}
